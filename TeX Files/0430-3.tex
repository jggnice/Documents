\documentclass[UTF8]{ctexart}
%中文论文类型
%%%%preamble 定义导言开始%%%%%%%%%%%
\usepackage{amsmath}
\usepackage{listings}

\usepackage{float}
\usepackage{geometry}
\geometry{a4paper,centering,scale = 0.6}%%页面尺寸
\usepackage{graphicx}%%导入pdf/png/eps
\title{C语言模拟实现内存管理}
\author{李世旺}
\date{\today}
\bibliographystyle{plain}%%声明参考文件格式
\newtheorem{thm}{定理}%%thm 变量
\newenvironment{myquote}{\begin{quote}\zihao{-6}\kaishu}{\end{quote}}
\newcommand\degree{^\circ}

%%%%preamble 定义导言结束%%%%%%%%%%%
\begin{document}
\maketitle
%制作标题
\begin{abstract}
内存作为计算机的一项重要资源,应该要合理的进行管理。
本文就连续分配存储管理方式中的各种动态分区分配方式进行比较。
并用C语言进行模拟实验,测算不同分配方式的差异。
\end{abstract}
%\begin{keywords}
%内存管理\sep连续分配\sep动态分区
%\end{keywords}
\newpage

\tableofcontents
%制作目录
\newpage
\section{实验目的}
用C语言模拟实现内存的动态分区分配管理
\section{实验原理与方案}
本实验中的动态分区分配又称为可变分区分配,
它是根据进程的实际需要,动态的为之分配内存空
间。在实现动态分区分配时将涉及到分区分配中所用的
数据结构、分区分配算法和分区的分配与回收操作这样三方面的问题。
\subsection{数据结构}
空闲分区表:用一个带头结点的单链表来存储,结构体的成员包括分区起始地址、分区大小、分区空闲时刻、下一个空闲分区地址。

繁忙分区表:类似空闲分区表,用一个带头结点的单链表来存储,结构体的成员包括分区起始地址、分区大小、分区空闲时刻、下一个繁忙分区地址。
\subsection{分区分配算法}
首次适应法:每次从空闲分区表的头部开始,找到可以分配的分区就分配。

循环首次适应法:分配分区的时候,从上次分配的地方开始往后面查找,到后面没有找到时,再从头开始查找。
\subsection{分区分配操作}
分配内存:如果请求的分区与目标分区相差不大,则直接分配出去,否则划分成两部分。同时还要修改空闲分区表和繁忙分区表。

回收内存:在分配分区的时候,繁忙分区成员的分区空闲时刻已经设置好了,时间触发回收操作。根据回收区的首地址来查找它在空闲分区的插入位置。
如果回收区与前一个空闲分区相连,或者与后一个空闲分区相连,此时应该尽可能扩大分区。
\section{执行结果与分析}
%%%%%%
对于事先给定的10000个任务数据,限制内存为 2147483648,
首次适应法用时:641.
循环首次适应法用时:642.
从结果分析来看,分配算法之间没有明显差异。

事实上,首次适应法会使得低地址部分的空间细分成小碎片,而为了找到大分区,需要费力地往后面查找。

循环首次适应法或许解决了分区切割不均匀的问题,但是这样可能会缺乏大的分区。

像最坏适应算法,它的分配最为均匀了,对于中小规模的任务,产生碎片的可能性最小。

和最坏适应算法相反的最佳适应算法,则会留下许许多多的碎片。
\section{详细代码}
\begin{lstlisting}[language = C]
%\begin{verbatim}
#include<stdio.h>
#include<stdlib.h>
#define GAP 5
typedef struct cnode
{
    unsigned int size;
    unsigned int begin;
    unsigned int finish;
    struct cnode * next;
} charnode;
unsigned int currenttime = 0;
charnode *ProcessList;
charnode *CycleFirstFit;
charnode *allocate(charnode *List, int runtime, int memsize)
{
    charnode *q, *p, *r, *last;
    p = List->next;
    if (NULL == p)
    {
        return NULL;
    }
    last = List;
    while (NULL != p)
    {
        if (p->size < memsize)
        {
            //    unsigned int a = 5,b = 6;
            //    printf("\n%d",a>b);
            //    printf("\n%d",a-b>0);
            last = p;
            p = p->next;
            continue;
        }
        else if (p->size - memsize <= GAP)
        {
            //////////////////////////////////
            /////////////////////////////////
            last->next = p->next;
            //////////////////////////////
            p->finish = currenttime + runtime;
            r = ProcessList->next;
            ProcessList->next = p;
            p->next = r;
            return p;
        }
        else
        {
            ///////////////////////
            q = (charnode *) malloc(sizeof(charnode));
            q->begin = p->begin;
            q->size = memsize;
            q->finish = currenttime + runtime;
            ///////////////////////
            r = ProcessList->next;
            ProcessList->next = q;
            q->next = r;
            ///////////////////////
            p->begin = p->begin + memsize;
            p->size = p->size - memsize;
            return q;
        }
    }            //;
    return NULL;
}
charnode *allocate2(charnode *List, int runtime, int memsize)
{
    int cycleflag = 1;
    charnode *q, *p, *r, *last;
    p = CycleFirstFit->next;
    if (NULL == p)
    {
        cycleflag = 1;
    }
    last = List;
    while (NULL != p)
    {
        if (p->size < memsize)
        {
            //    unsigned int a = 5,b = 6;
            //    printf("\n%d",a>b);
            //    printf("\n%d",a-b>0);
            last = p;
            p = p->next;
            continue;
        }
        else if (p->size - memsize <= GAP)
        {
            //////////////////////////////////
            /////////////////////////////////
            last->next = p->next;
            //////////////////////////////
            p->finish = currenttime + runtime;
            r = ProcessList->next;
            ProcessList->next = p;
            p->next = r;
            CycleFirstFit = last;
            cycleflag = 0;
            return p;
        }
        else
        {
            ///////////////////////
            q = (charnode *) malloc(sizeof(charnode));
            q->begin = p->begin;
            q->size = memsize;
            q->finish = currenttime + runtime;
            ///////////////////////
            r = ProcessList->next;
            ProcessList->next = q;
            q->next = r;
            ///////////////////////
            p->begin = p->begin + memsize;
            p->size = p->size - memsize;
            CycleFirstFit = p;
            cycleflag = 0;
            return q;
        }
    }            //;
    if(!cycleflag)return List;
    /////////////////////////////////////////////////////////
    p = List->next;
    if (NULL == p)
    {
        return NULL;
    }
    last = List;
    while (NULL != p)
    {
        if (p->size < memsize)
        {
            //    unsigned int a = 5,b = 6;
            //    printf("\n%d",a>b);
            //    printf("\n%d",a-b>0);
            last = p;
            p = p->next;
            continue;
        }
        else if (p->size - memsize <= GAP)
        {
            //////////////////////////////////
            /////////////////////////////////
            last->next = p->next;
            //////////////////////////////
            p->finish = currenttime + runtime;
            r = ProcessList->next;
            ProcessList->next = p;
            p->next = r;
            CycleFirstFit = last;
            return p;
        }
        else
        {
            ///////////////////////
            q = (charnode *) malloc(sizeof(charnode));
            q->begin = p->begin;
            q->size = memsize;
            q->finish = currenttime + runtime;
            ///////////////////////
            r = ProcessList->next;
            ProcessList->next = q;
            q->next = r;
            ///////////////////////
            p->begin = p->begin + memsize;
            p->size = p->size - memsize;
            CycleFirstFit = p;
            return q;
        }
    }            //;
    return NULL;
}

void deallocate(charnode *List, charnode *node)
{
    //////////////////////////////////////////
    charnode *p = List->next, *q, *r, *last;
    if (NULL == p)
    {
        ///////////////////////
        List->next = node;
        node->next = p;
        return;
    }
    last = List;
    while (NULL != p)
    {
        q = p->next;
        if (p->begin + p->size == node->begin)
        {
            if (NULL == q)
            {
                //////////   node insert after p    /////////////
                p->size += node->size;
                return;
            }
            if (node->begin + node->size == q->begin)
            {
                //////////   node insert after p    /////////////
                r = q->next;
                p->next = r;
                p->size += (node->size + q->size);
                return;
            }
            else
            {
                //////////   node insert after p    /////////////
                p->size += node->size;
                return;
            }
        }
        else if (p->begin < node->begin)
        {
            last = p;
            p = p->next;
            continue;
        }
        else
        {
            //////////   node insert before p    /////////////
            if (node->begin + node->size == p->begin)
            {
                last->next = node;
                node->next = q;
                node->size += p->size;
                return;
            }
            else
            {
                last->next = node;
                node->next = p;
                return;
            }
        }
    }            // node insert at the end of the list
    last->next = node;
    node->next = p;
    return;
}
int main(int argc, char** argv)
{
    unsigned int i;
    unsigned int num_of_wait;
    unsigned int max_memory;
    unsigned int num_of_line;
    FILE *fp;
    if (2 != argc)
        printf("\nTips: a.exe data.txt");
    fp = fopen(argv[1], "rt");
    fscanf(fp, "%u", &num_of_line);
    fscanf(fp, "%u", &max_memory);
    unsigned int cursor = 0;
    unsigned int triggercursor = 0;
    unsigned int trigger[num_of_line];
    unsigned int visited[num_of_line];
    unsigned int arraive[num_of_line];
    unsigned int runtime[num_of_line];
    unsigned int memsize[num_of_line];
    printf("\n line of data : %u\n limit of memory : %u", num_of_line, max_memory);
//    unsigned int a = 5,b = 6;
//    printf("\n%d",a>b);
//    printf("\n%d",a-b>0);
//    return 0;
    trigger[triggercursor] = 0;
    for (i = 0; i < num_of_line; i++)
    {
        fscanf(fp, "%u%u%u", &arraive[i], &runtime[i], &memsize[i]);
        visited[i] = 0;
        if (trigger[triggercursor] != arraive[i])
        {
            trigger[++triggercursor] = arraive[i];
        }
    }
    trigger[++triggercursor] = 100000;
    charnode *FreeList, *node;
    ProcessList = (charnode *) malloc(sizeof(charnode));
    ProcessList->next = NULL;
    FreeList = (charnode *) malloc(sizeof(charnode));
    FreeList->next = NULL;
    ///////////////////////
    node = (charnode *) malloc(sizeof(charnode));
    node->begin = 0;
    node->size = max_memory;
    node->finish = 0;
    deallocate(FreeList, node);
    CycleFirstFit = FreeList;
    //////////////////////////
    charnode* (*all)(charnode *List, int runtime, int memsize);
    printf("\ninput 1 for First Fit or 2 for Next Fit : -->");
    int exitflag = 1;    
    while(exitflag)
    {
        char ch = getchar();
        switch(ch)
        {
            case '1': 
            {
                all = allocate;
                exitflag = 0;
                break;
            }
            case '2': 
            {
                all = allocate2;
                exitflag = 0;
                break;
            }
            default : 
            {
                printf("\ninput 1 for First Fit or 2 for Next Fit : -->");
                break;
            }
        }
        
    }
    printf("\n_____wait for a minute ...");
    while (1)
    {
        //////// Time Machine Real Trigger : when process arraived
        //////// Time Machine Trigger : when process finished
        num_of_wait = 0;
        for (i = 0; i < num_of_line && arraive[i] <= currenttime; i++)
        {
            if (!visited[i])
            {
                num_of_wait++;
            }
        }
        if (!num_of_wait)
        {
            if (num_of_line - 1 <= i)
            {
                charnode *p = ProcessList->next;
                int maxtime = 0;
                if (NULL == p)
                {
                    printf("\n_____well done !_____");
                    printf("\n the answer : %u", currenttime);
                    return 0;
                }
                while (NULL != p)
                {
                    if (maxtime < p->finish)
                    {
                        maxtime = p->finish;
                    }
                    p = p->next;
                }
                printf("\n_____well done !_____");
                printf("\n the answer : -> %u", maxtime);
                return 0;
            }
            else
            {
                currenttime = arraive[i + 1];
            }
        }
        //////////////////////////
        for (i = 0; i < num_of_line && arraive[i] <= currenttime; i++)
        {
            if (!visited[i])
            {
                if (NULL != all(FreeList, runtime[i], memsize[i]))
                    visited[i] = 1;
            }
        }
        /////////////////////////
        charnode *p = ProcessList->next;
        charnode *last = ProcessList;
        charnode *r, *s;
        int mintime = 1000000;
        while (NULL != p)
        {
            if (mintime > p->finish)
            {
                mintime = p->finish;
                node = last;
            }
            {
                last = p;
                p = p->next;
            }
        }
        //////////////////////////////////////////
        if (NULL != ProcessList->next)
        {
            // |------------>
            // |   --->
            // |---------->
            r = node->next;
            s = r->next;
//            printf("\n%u", r->finish);
            if (trigger[cursor] < r->finish && currenttime <= trigger[cursor])
            {
                currenttime = trigger[cursor++];
                continue;
            }
            // confirm to recycle
            node->next = s;
            deallocate(FreeList, r);
            currenttime = r->finish;
            cursor = 0;
            while(trigger[cursor] < currenttime && cursor < triggercursor)
            {
                cursor++;
            }
        }
        ///////////////////////////////////
    }
}


%\end{verbatim}
\end{lstlisting}
\end{document}