\documentclass[UTF8,a4paper,12pt]{ctexart}
%中文论文类型
%%%%preamble 定义导言开始%%%%%%%%%%%
\usepackage{indentfirst,latexsym,bm}
\setlength{\parindent}{2em} %段首缩进量
\renewcommand{\baselinestretch}{1.2} %行距倍数
\renewcommand{\abstractname}{摘\quad要} %重置变量值

\usepackage[colorlinks]{hyperref} %目录超链接
\usepackage{amsmath}
\usepackage{listings}

\usepackage{float}
\usepackage{geometry}
\geometry{a4paper,centering,scale = 0.6}%%页面尺寸

\usepackage{graphicx}%%导入pdf/png/eps

\title
{
C语言读取ext2文件系统
\thanks{supported by NSFC.}
\footnotetext{\textit{Keyword:}特征值}
}
\author
{李世旺\\[5pt] Departmrnt of Mathematics, Soochow University\\(苏州大学数学系)}
\date{\today}
\bibliographystyle{plain}%%声明参考文件格式
\newtheorem{thm}{定理}%%thm 变量
\newenvironment{myquote}{\begin{quote}\zihao{-6}\kaishu}{\end{quote}}
\newcommand\degree{^\circ}

%%%%preamble 定义导言结束%%%%%%%%%%%
\begin{document}
\maketitle
%制作标题
\begin{abstract}
ext2作为简单的文件系统,我们应该学习它的存储格式。
本文就ext2文件系统进行分析并用C语言读取文件。
\end{abstract}
%\begin{keywords}
%内存管理\sep连续分配\sep动态分区
%\end{keywords}
\newpage

\tableofcontents
%制作目录
\newpage
\section{实验目的}
用C语言读取ext2文件系统
\section{实验原理与方案}
ext2文件系统的基本单元是一个个的block,每个block常常是 1KB、2KB、4KB等。文件系统的最前 1KB 留做启动用途,1KB 到 2KB 的地方
是 superblock 的所在。下一个 block 开始就是 块群描述表。我们把一定量的 block 分割开来,叫做块群。
\subsection{inode}
每一个目录或文件都有一个 inode 指定。
根目录的 inode 一般为 2, 也就是第二项 inode.
\subsection{目录文件}
每一个目录或文件都有一个 inode 指定。
目录看作是一个文件,这个文件包含目录下的所有文件或目录的名字和权限等属性,这些属性所
\section{执行结果与分析}
%%%%%%
$$R^e = R \otimes R$$
\begin{verbatim}

int main(int argc, char** argv)
{
    Initial(argc, argv);
    return 0;
}


\end{verbatim}
%\end{lstlisting}
\begin{center}
\setlength{\unitlength}{1mm}
\begin{picture}(60,30)
\linethickness{1pt}
\put(0,0){\vector(1,0){60}}
\put(0,0){\vector(0,1){30}}
\thicklines
\put(5,7){\line(5,2){50}}
%\qbezier(5,7)(10,9)(55,27)
\thinlines
\multiput(5,7)(3,0){15}{\line(1,0){2}}
\multiput(50,7.00)(0,3){6}{\line(0,1){2}}
\put(28,3){$\Delta x$}
\put(51,14){$\Delta y$}
\circle*{50}
\end{picture}
\qquad
\begin{picture}(60,30)
\linethickness{1pt}
\put(0,0){\vector(1,0){60}}
\put(0,0){\vector(0,1){30}}
\thicklines
%\put(5,7){\line(5,2){50}}
\qbezier[1000](5,7)(10,9)(55,27)
\thinlines
\multiput(5,7)(3,0){15}{\line(1,0){2}}
\multiput(50,7.00)(0,3){6}{\line(0,1){2}}
\put(28,3){$\Delta x$}
\put(51,14){$\Delta y$}
\circle*{50}
\end{picture}
\end{center}
\begin{center}
\setlength{\unitlength}{1mm}
\begin{picture}(60,50)
\put(20,7){\circle*{2}}
\put(20,7){\circle{14}}
\put(0,11){\line(6,1){35}}
\put(15.5,13.6){\line(3,5){4.5}}
\put(20,21){\line(1,-2){3}}
\put(20,21){\line(0,1){7}}
\put(13,28){\line(6,-1){14}}
\put(20,30){\circle{4}}
\end{picture}

\end{center}
\begin{center}
\setlength{\unitlength}{1mm}
\begin{picture}(60,50)
\thicklines
\put(20,11){\oval(20,22)}
\multiput(16,15)(8,0){2}{\oval(6,4)[t]}
\put(20,9){\oval(2,2)[b]}
\put(20,5){\oval(6,5)[b]}
\end{picture}

\end{center}
\begin{center}
\setlength{\unitlength}{2mm}
\begin{picture}(30,25)
\put(0,0){\circle*{1}}\put(0,20){\circle*{1}}\put(40,20){\circle*{1}}\put(40,0){\circle*{1}}
\qbezier[20](0,0)(0,10)(0,20)\qbezier[40](0,20)(20,20)(40,20)
\qbezier(0,0)(0,20)(40,20)\qbezier(0,0)(40,0)(40,20)
\qbezier[20](40,0)(40,10)(40,20)\qbezier[40](0,0)(20,0)(40,0)
\put(1,0){\framebox(0,0)[bl]{(0,0)}}\put(1,19){\framebox(0,0)[tl]{(0,20)}}
\put(40,18.5){\framebox(0,0)[tr]{(0,0)}}\put(40,1){\makebox(0,0)[br]{(0,0)}}
\end{picture}

\end{center}
\end{document}